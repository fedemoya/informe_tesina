\section{Modelica}

\begin{frame}{Modelica}
    \begin{block}<1->{}
        Es un lenguaje de modelado (estándar abierto) que permite la especificación de modelos matemáticos de complejos sistemas naturales o artificiales.
    \end{block}
    Características principales:
    \begin{itemize}
        \item<2-> Orientado a objetos.
        \item<3-> Basado en ecuaciones.
        \note{
            Esto permite un modelado acausal lo cual facilita la reutilización de las clases ya que las ecuaciones no especifican una dirección particular para el flujo de datos. De esta manera una clase Modelica se puede adaptar a contextos con diferentes flujos de datos.
        }
        \item<4-> Permite describir y conectar componentes de modelos pertenecietes a diferentes dominios.
    \end{itemize}
\end{frame}

\begin{frame}{Conceptos básicos}
    \begin{itemize}
        \item<1-> Un modelo Modelica se construye a partir de clases.
        \item<2-> Los principales elementos que contiene una clase  son declaraciones de variables y ecuaciones.
    \end{itemize}
\end{frame}

\begin{frame}[fragile]{Conceptos básicos}
   \framesubtitle{Ejemplo: Circuito RLC}
   \fontsize{8pt}{7.2}\selectfont
    \begin{columns}
        \begin{column}{0.4\textwidth}
            \begin{eqnarray*}
            u_{0} & = & \sin\left(t\right)\\
            u_{1} & = & R_{1}i_{1}\\
            u_{2} & = & R_{2}i_{2}\\
            u_{L} & = & L\frac{di_{L}}{dt}\\
            i_{C} & = & C\frac{du_{C}}{dt}\\
            u_{0} & = & u_{1}+u_{2}\\
            u_{L} & = & u_{1}+u_{2}\\
            u_{C} & = & u_{2}\\
            i_{0} & = & i_{1}+i_{L}\\
            i_{1} & = & i_{2}+i_{C}
            \end{eqnarray*}
        \end{column}
        \begin{column}{0.6\textwidth}
            \begin{lstlisting}[language=Modelica]
model RLC
    Real u1; Real u2;
    Real uL; Real iC;
    Real i1; Real i2;
    Real u0; Real i0;
    Real iL; Real uC;
    parameter Real R1;
    parameter Real R2;
    parameter Real L;
    parameter Real C;
    equation
        u0 = sin(time);
        u1 = R1 * i1;
        u2 = R2 * i2;
        uL = L * der(iL);
        iC = C *der(uC);
        u0 = u1 + uC;
        uL = u1 + u2;
        uC = u2;
        i0 = i1 + iL;
        i1 = i2 + iC;
end RLC;
            \end{lstlisting}
        \end{column}
    \end{columns}
\end{frame}

\begin{frame}[fragile]{Modelado acausal}
    La ecuación característica de una resistencia, $Ri=v$, puede utilizarse de diferentes maneras.
    \pause
    \begin{itemize}
        \item<2-> Para calcular la corriente a partir del voltaje y de la resistencia: $i :=\frac{v}{R}$
        \item<3-> Para calcular el voltaje a partir de la resistencia
y de la corriente: $v :=Ri$
        \item<4-> Para calcular la resistencia a partir del voltaje y
de la corriente: $R :=\frac{v}{i}$
    \end{itemize}
\end{frame}
