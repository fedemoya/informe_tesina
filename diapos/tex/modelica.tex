\section{Modelica}

\begin{frame}{Modelica}
    \begin{block}<1->{}
        Es un lenguaje de modelado (estándar abierto) que permite la especificación de modelos matemáticos de complejos sistemas naturales o artificiales.
    \end{block}
    Características principales:
    \begin{itemize}
        \item<2-> Modelica es un lenguaje orientado a objetos con un concepto general de clase que unifica las ideas de clases, tipos genéricos y subtipos en un solo lenguaje. Esto facilita la reutilización de componentes y la evolución de los modelos. 
    \end{itemize}
\end{frame}

\begin{frame}{Modelica}
    \begin{itemize}
        \item<1-> Modelica está principalmente basado en ecuaciones en lugar de asignaciones. Esto permite un modelado acausal lo cual facilita la reutilización de las clases ya que las ecuaciones no especifican una dirección particular para el flujo de datos. De esta manera una clase Modelica se puede adaptar a contextos con diferentes flujos de datos.
        \item<2-> Modelica permite describir y conectar componentes de modelos pertenecientes a diferentes dominios como eléctricos, mecánicos, termodinámica, hidráulica, biología, control, etc. 
    \end{itemize}
\end{frame}

\begin{frame}{Conceptos básicos}
    \begin{itemize}
        \item<1-> Un modelo Modelica se construye a partir de clases.
        \item<2-> Los principales elementos que contiene una clase  son declaraciones de variables y ecuaciones.
    \end{itemize}
\end{frame}

\begin{frame}[fragile]{Conceptos básicos}
   \framesubtitle{Ejemplo: Péndulo Plano}
   \fontsize{8pt}{7.2}\selectfont
    \begin{columns}
    % \begin{minipage}[t]{2cm}
        \begin{column}{0.4\textwidth}
                \begin{align*}
                    m\dot{v_{x}}  & =-F\sin(\varphi)\nonumber \\
                    m\dot{v_{y}}  & =-F\cos(\varphi)-mg\nonumber \\
                    \dot{x}       & =v_{x}\nonumber \\
                    \dot{y}       & =v_{y}\\
                    \dot{\varphi} & =\omega\nonumber \\
                    x             & =L\sin(\varphi)\nonumber \\
                    y             & =-L\cos\left(\varphi\right)\nonumber
                \end{align*}
        \end{column}
        % \end{minipage}
        \begin{column}{0.6\textwidth}
        % \begin{minipage}[t]{10cm}
            \begin{lstlisting}[language=Modelica]
class Pendulum "Planar Pendulum"
  parameter Real m=1, g=9.81, L=O.5;
  Real F, phi, omega;
  output Real x(start=O.5);
  output Real y(start=O);
  output Real vx,vy;
  equation
    m*der(vx)=-F*sin(phi);
    m*der(vy)=-F*cos(phi)-m*g;
    der(x)=vx;
    der(y)=vy;
    der(phi)=omega;
    x=L*sin(phi);
    y=-L*cos(phi);
end Pendulum;
            \end{lstlisting}
        % \end{minipage}
        \end{column}
    \end{columns}
\end{frame}

\begin{frame}[fragile]{Modelado acausal}
    La ecuación característica de una resistencia, $Ri=v$, puede utilizarse de diferentes maneras.
    \pause
    \begin{itemize}
        \item<1-> Para calcular la corriente a partir del voltaje y de la resistencia: $i :=\frac{v}{R}$
        \item<2-> Para calcular el voltaje a partir de la resistencia
y de la corriente: $v :=R\times i$
        \item<3-> Para calcular la resistencia a partir del voltaje y
de la corriente: $R :=\frac{v}{i}$
    \end{itemize}
\end{frame}

\begin{frame}{Compilación y Ejecución de Modelos Modelica}
    \note{Análisis sintáctico del código fuente Modelica del cual se obtiene un árbol sintáctico abstracto. Se analiza gramaticalmente, se realiza la verificación de tipos, las clases se heredan y expanden, las ecuaciones connect se convierten en ecuaciones comunes,
etc. El resultado de este proceso de análisis y traducción es un conjunto plano de ecuaciones, constantes, variables y funciones. No queda ningún rastro de la estructura de objetos.
Luego un módulo de optimización compuesto por algoritmos de simplificación algebraica, métodos de
reducción de índices, etc., elimina la mayoría de las ecuaciones dejando solo un conjunto mínimo que eventualmente va a ser resuelto numéricamente. Por ejemplo, si dos variables son sintácticamente equivalentes solo
se conserva una.
Causalización.
Finalmente se genera código en algún lenguaje de programación convencional,
usualmente C, y se lo enlaza con algún método de integración numérica.
}
    \begin{figure}[H]
      \centering
      \includegraphics[width=1\textwidth]{graphics/compilacion_ejecucion_modelica.png}
    \end{figure}
\end{frame}
