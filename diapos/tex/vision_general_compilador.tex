\section{Visión General del Compilador ModelicaCC}

\begin{frame}[fragile]{El Compilador ModelicaCC}
    \note{
        La compilación de un modelo Modelica es el proceso mediante el cual se transforma un modelo de alto nivel, el cual puede contener clases y ecuaciones acausales, en un modelo plano, sin clases, y donde las ecuciones se encuentran ordenadas o causalizadas.\\
        El compilador ModelicaCC se planteo con una arquitectura de pipeline o de tubos y filtros. Es decir se compone de aplicaciones independientes cada una de las cuales recibe como entrada un modelo Modelica y produce como salida otro modelo Modelica, modificado de alguna manera pero equivalente al que se recibio como entrada.
    }
    \begin{itemize} 
        \item<1-> Desarrollado en C++. 
        \item<2-> Implementa una arquitectura de pipeline o tubos y filtros.
    \end{itemize}
    \pause
    \begin{center}
        \includegraphics[width=\textwidth]{graphics/compilacion_modelica.png}
    \end{center}
    \begin{itemize} 
        \item<4-> flatter: Es el componente encargado de el análisis sintáctico y semántico, y del aplanado de la estructura de objetos. 
        \item<5-> mmo: Es el componente responsable de reescribir el modelo utilizando un subconjunto del lenguaje Modelica denominado μModelica.
        \item<6-> antialias: Es el componente encargado de eliminar ecuaciones triviales de la forma \verb+a=b+.
        \item<7-> cauzalize: Es el componente responsable de la causalización del conjunto de ecuaciones del modelo.
    \end{itemize}
\end{frame}