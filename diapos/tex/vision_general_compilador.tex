\section{Visión General del Compilador ModelicaCC}

\begin{frame}
    La compilación de un modelo Modelica es el proceso mediante el cual se transforma un modelo de alto nivel, el cual puede contener clases y ecuaciones acausales, en un modelo plano, sin clases, y donde las ecuciones se encuentran ordenadas o causalizadas.
    \pause
    El compilador ModelicaCC se planteo como un conjunto de componentes independientes.
    \pause
    \begin{itemize} 
        \item<1-> \textbf{flatter}: Es el componente encargado de el análisis sintáctico y semántico, y del aplanado de la estructura de objetos. 
        \item<2-> \textbf{μModelica}: Es el componente responsable de reescribir el modelo utilizando un subconjunto del lenguaje Modelica denominado μModelica.
        \item<3-> \textbf{antialias}: Es el componente encargado de eliminar ecuaciones triviales de la forma \verb+a=b+.
        \item<4-> \textbf{cauzalize} Es el componente responsable de la causalización del conjunto de ecuaciones del modelo.
\end{frame}